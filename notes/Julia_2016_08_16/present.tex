\documentclass{beamer}
\mode<presentation>{
  \usetheme{Boadilla}
  \usefonttheme[onlylarge]{structurebold}
  \usefonttheme[stillsansseriflarge]{serif}
  \setbeamerfont*{frametitle}{size=\normalsize,series=\bfseries}
  % \setbeamertemplate{navigation symbols}{}
  \setbeamercovered{transparent}
}
\usepackage[english]{babel}
% \usepackage[latin1]{inputenc}
\usepackage[BoldFont,SlantFont,CJKnumber,fallback]{xeCJK}
\setmainfont{DejaVu Serif}
\setsansfont{DejaVu Sans}
\setmonofont{DejaVu Sans Mono}
\setCJKmainfont{Droid Sans Fallback}
\setCJKsansfont{Droid Sans Fallback}
\usepackage{times}
\usepackage[T1]{fontenc}
\usepackage{amsmath}
\usepackage{amssymb}
\usepackage{esint}
\usepackage{hyperref}
\usepackage{tikz}
\usepackage{xkeyval}
\usepackage{xargs}
\usepackage{verbatim}
\usepackage{listings}
\usepackage{multimedia}
\usepackage{pgfplots}
\usepgfplotslibrary{colormaps}
\usetikzlibrary{
  arrows,
  calc,
  decorations.pathmorphing,
  decorations.pathreplacing,
  decorations.markings,
  fadings,
  positioning,
  shapes
}

\pgfdeclareradialshading{glow}{\pgfpoint{0cm}{0cm}}{
  color(0mm)=(white);
  color(3mm)=(white);
  color(7mm)=(black);
  color(10mm)=(black)
}
\pgfdeclareverticalshading{beam}{2cm}{
  % manual 1082-1083; later - shading is assumed to be 100bp diameter ??
  color(0cm)=(black);
  color(0.3cm)=(black);
  color(1cm)=(white);
  color(1.7cm)=(black);
  color(2cm)=(black)
}

\begin{tikzfadingfrompicture}[name=glow fading]
  \shade [shading=glow] (0,0) circle (1);
\end{tikzfadingfrompicture}

\begin{tikzfadingfrompicture}[name=beam fading]
  \shade [shading=beam] (-1,-1) rectangle (1, 1);
\end{tikzfadingfrompicture}

% not mandatory, but I though it was better to set it blank
\setbeamertemplate{headline}{}
\def\beamer@entrycode{\vspace{-\headheight}}

\tikzstyle{snakearrow} = [decorate, decoration={pre length=0.2cm,
  post length=0.2cm, snake, amplitude=.4mm,
  segment length=2mm},thick, ->]

%% document-wide tikz options and styles

\tikzset{%
  % >=latex, % option for nice arrows
  inner sep=0pt,%
  outer sep=2pt,%
  mark coordinate/.style={inner sep=0pt,outer sep=0pt,minimum size=3pt,
    fill=black,circle}%
}
\tikzset{
  % Define standard arrow tip
  >=stealth',
  % Define style for boxes
  punkt/.style={
    rectangle,
    rounded corners,
    draw=black, very thick,
    text width=8em,
    minimum height=2.5em,
    text centered},
}
\makeatletter
\newbox\@backgroundblock
\newenvironment{backgroundblock}[2]{%
  \global\setbox\@backgroundblock=\vbox\bgroup%
  \unvbox\@backgroundblock%
  \vbox to0pt\bgroup\vskip#2\hbox to0pt\bgroup\hskip#1\relax%
}{\egroup\egroup\egroup}
\addtobeamertemplate{background}{\box\@backgroundblock}{}
\makeatother

\title{Julia 底层优化}
\author{俞颐超\\yyc1992@gmail.com}
\institute[\footnotesize {@yuyichao}]{\footnotesize {@yuyichao}\\\includegraphics[width=1cm]{yuyichao.jpg}}
\date{Aug 16, 2016}

\begin{document}

\begin{frame}
  \titlepage
\end{frame}

\begin{frame}{}
  \tableofcontents
\end{frame}

\section{垃圾收集基础}
\begin{frame}{垃圾收集}
  \only<+>{
    \begin{block}{垃圾收集的作用}
      \begin{itemize}
      \item 分配内存
      \item 清理垃圾\\
        一般情况下大部分分配的内存都是垃圾
      \end{itemize}
    \end{block}
  }
  \only<+>{
    \begin{block}{Mark-Sweep}
      \begin{itemize}
      \item Mark\\
        分辨哪些内存是有用的
      \item Sweep\\
        回收内存
      \end{itemize}
    \end{block}
  }
  \only<+>{
    \begin{block}{减少延迟}
      \begin{itemize}
      \item 分代
      \item 并行 (parallel)
      \item 并行 (concurrent)
      \end{itemize}
    \end{block}
  }
\end{frame}

\section{多线程}
\begin{frame}{多线程}
  \only<+>{
    \begin{block}{线程同步开销}
      \begin{itemize}
      \item 锁
      \item 内存屏障 (Memory barrier)
      \item 线程局部变量
      \end{itemize}
    \end{block}
  }
  \only<+>{
    \begin{block}{多线程垃圾收集支持}
      \begin{itemize}
      \item Stop the world
      \item 安全点 (safepoint)
      \item GC 状态
      \end{itemize}
    \end{block}
  }
\end{frame}

\section{调试内存错误}
\begin{frame}{调试内存错误}
  \only<+>{
    \begin{block}{常见内存错误原因}
      \begin{itemize}
      \item 越界访问
      \item ccall GC 引用
      \end{itemize}
    \end{block}
  }
  \only<+>{
    \begin{block}{调试方法}
      \begin{itemize}
      \item ASLR
      \item rr
      \end{itemize}
    \end{block}
  }
\end{frame}

\end{document}
